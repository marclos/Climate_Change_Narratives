\documentclass{article}\usepackage[]{graphicx}\usepackage[]{color}
%% maxwidth is the original width if it is less than linewidth
%% otherwise use linewidth (to make sure the graphics do not exceed the margin)
\makeatletter
\def\maxwidth{ %
  \ifdim\Gin@nat@width>\linewidth
    \linewidth
  \else
    \Gin@nat@width
  \fi
}
\makeatother

\definecolor{fgcolor}{rgb}{0.345, 0.345, 0.345}
\newcommand{\hlnum}[1]{\textcolor[rgb]{0.686,0.059,0.569}{#1}}%
\newcommand{\hlstr}[1]{\textcolor[rgb]{0.192,0.494,0.8}{#1}}%
\newcommand{\hlcom}[1]{\textcolor[rgb]{0.678,0.584,0.686}{\textit{#1}}}%
\newcommand{\hlopt}[1]{\textcolor[rgb]{0,0,0}{#1}}%
\newcommand{\hlstd}[1]{\textcolor[rgb]{0.345,0.345,0.345}{#1}}%
\newcommand{\hlkwa}[1]{\textcolor[rgb]{0.161,0.373,0.58}{\textbf{#1}}}%
\newcommand{\hlkwb}[1]{\textcolor[rgb]{0.69,0.353,0.396}{#1}}%
\newcommand{\hlkwc}[1]{\textcolor[rgb]{0.333,0.667,0.333}{#1}}%
\newcommand{\hlkwd}[1]{\textcolor[rgb]{0.737,0.353,0.396}{\textbf{#1}}}%
\let\hlipl\hlkwb

\usepackage{framed}
\makeatletter
\newenvironment{kframe}{%
 \def\at@end@of@kframe{}%
 \ifinner\ifhmode%
  \def\at@end@of@kframe{\end{minipage}}%
  \begin{minipage}{\columnwidth}%
 \fi\fi%
 \def\FrameCommand##1{\hskip\@totalleftmargin \hskip-\fboxsep
 \colorbox{shadecolor}{##1}\hskip-\fboxsep
     % There is no \\@totalrightmargin, so:
     \hskip-\linewidth \hskip-\@totalleftmargin \hskip\columnwidth}%
 \MakeFramed {\advance\hsize-\width
   \@totalleftmargin\z@ \linewidth\hsize
   \@setminipage}}%
 {\par\unskip\endMakeFramed%
 \at@end@of@kframe}
\makeatother

\definecolor{shadecolor}{rgb}{.97, .97, .97}
\definecolor{messagecolor}{rgb}{0, 0, 0}
\definecolor{warningcolor}{rgb}{1, 0, 1}
\definecolor{errorcolor}{rgb}{1, 0, 0}
\newenvironment{knitrout}{}{} % an empty environment to be redefined in TeX

\usepackage{alltt}

\title{How to Export Files for Sakai}
\IfFileExists{upquote.sty}{\usepackage{upquote}}{}
\begin{document}
\maketitle

\section{Text to Introduce your Blog}

For each blog, we'll need a introductory link -- text that encourages a reader to look at your blog from the introduction that I have outline. Create a file ('File/New/Text File') called `hook' and put some text in there that can be used in the introduction and I'll create a link to your blog. 

When submiting to the Sakai site, please include the `hook' in the exported files.  

\section{Naming your Blog}

Please name your blog with your surname only, e.g. `LosHuertos.Rmd', then I can make links from the introduction you have written to your blog in a relatively straighforward way. 

\section{Cleaning up Directory}

\begin{figure}
\includegraphics[width=\textwidth]{FourWindows}
\caption{Start by selecting the `Files' tab in the lower, right pane.}
\end{figure}

I suggest you clean up your directory, so you only have files that are useful to your blog; thus reducing the cluttter. However, I rarely delete files. Instead, I create a new folder call `OLD' and move files in there that I am not using.

\begin{figure}
\includegraphics[width=\textwidth]{CreateFolder}
\caption{Creating the folder is easy, but relies on using the little gear icon in the files pane.}
\end{figure}

In the lower right window, natigate to the `Files' tab. Click on `New Folder' and name it `OLD'. Now select on the files using the little boxes on the left and navigate the the 'More' menu that has the little gear icon. Select `Move' and choose the `OLD' to move the files. Finally, knit to make sure nothhing has broken by accident during the move. 

\begin{figure}
\includegraphics[width=\textwidth]{MoveFiles}
\caption{Again, moving folders is pretty straightforward, if you can select then navigate to the gear icon.}
\end{figure}


Later, you can delete these old files too, but there's not hurry -- you might need to dredge code up from an old file, so hanging on to them can serve you later. 

\section{Exporting Files}

Similar to the moving process, you will select the files you want to export by checking the boxes on the left of the files and then navigate to the `More' menu item with the gear icon. Then select export and R studio will prompt you to determine the location you want to download the file.

\begin{figure}[h]
\includegraphics[width=\textwidth]{ExportFiles}
\caption{The export command creates a zip file that can downloaded to your local computer. Beware: Apple computers automatically unzip the files, which can't be uploaded to Rstudio. So, MAC users have to compress the files before uploading to Sakai.}
\end{figure}


Note: For PC users this is a zip file that can be uploaded directly to Sakai. However, for MAC users, you must compress the file so that it can be put into Sakai. 

I will then upload the files into the `docs' folder, where I will knit the files to create the html. These files will by synced and publicly visible in about 10-15 minutes after I upload them. 









\end{document}
