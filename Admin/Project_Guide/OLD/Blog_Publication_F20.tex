%\subsection{Publishing Revised Blog}

\subsubsection{Rational}

Our capacity to publish our blogs demonstrates that our projects have value beyond our classroom. In addition, these provide a litmus test for our work -- how will the public or specific stakeholders respond to our efforts. Will they see this a valueable, value-added, or problematic?  Although we might not get immediate feedback, the process to publish our blogs gives an opportunity that would be missing if we only wrote papers for the instructor!


\subsubsection{Assignment}

Capitalizing on the regional data analysis and impact summary, create a blog that describes the patterns of clilmate change and their implications. Your final products should include:

\begin{itemize*}
  \item Effectively display climate patterns from NOAA repositories, with at least 6 decades of data. Be sure all graphics are appropriate labeled and have captions that the reader can use to intrepret the data;
  \item Analyze the data using a linear model using R (i.e. lm);
  \item Describe the methods used to obtain and analyze the data; and
  \item Evaluate peer review literature to determine potential regional impacts from climate change -- be sure to include ecological and economic impacts; 
  \item Cite instances of how various scientific arguments have been distorted and politicized;
  \item Identify how conventional scientific standards have been compromised and how arguments that might be based on distortions can be countered.
\end{itemize*}

If it helps, read the Project\_Report.pdf on the Project Site for some helpful hints.

\subsubsection{Submission Format and Naming Convention}

The Blog will be published online (via \url{Github.com}), using the following naming convention: Lastname.Rmd and Lastname.html. 

\subsubsection{Published Blog Grading}

The Blogs will be grading using Table \ref{tab:bloggrading}. 

\begin{table}[h]
\caption{Climate Science Blog Grading.}
\label{tab:bloggrading}
\begin{tabular}{llll}\hline
Standard              &   DRAFT Percent   & Final \\ \hline\hline
Effective Figures             & 20\% &    & 10\% \\
Appropriate Trend Analysis    & 20\% &    & 10\% \\
Described Methods             & 20\% &    & 15\% \\
Peer-Reviewed Literature Effectively Discussed & 15\% & \%\\
Linkage to Climate Activists use and claims    & 50\% & \%\\
%Counter Arguments to Distorted Claims         & 20\% & \%\\
\hline
\end{tabular}
\end{table}

